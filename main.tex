%% start of file `template.tex'.
%% Copyright 2006-2013 Xavier Danaux (xdanaux@gmail.com).
%
% This work may be distributed and/or modified under the
% conditions of the LaTeX Project Public License version 1.3c,
% available at http://www.latex-project.org/lppl/.

% \usepackage[left=1in, right=1in, top=1in, bottom=1in]{geometry}

\documentclass[11pt,a4paper,sans]{moderncv}        % possible options include font size ('10pt', '11pt' and '12pt'), paper size ('a4paper', 'letterpaper', 'a5paper', 'legalpaper', 'executivepaper' and 'landscape') and font family ('sans' and 'roman')

% moderncv themes
\moderncvstyle{banking}                            % style options are 'casual' (default), 'classic', 'oldstyle' and 'banking'
\moderncvcolor{blue}                                % color options 'blue' (default), 'orange', 'green', 'red', 'purple', 'grey' and 'black'
%\renewcommand{\familydefault}{\sfdefault}         % to set the default font; use '\sfdefault' for the default sans serif font, '\rmdefault' for the default roman one, or any tex font name
\nopagenumbers{}                                  % uncomment to suppress automatic page numbering for CVs longer than one page

% character encoding
\usepackage[utf8]{inputenc}    
\usepackage{pdfpages}%
% if you are not using xelatex ou lualatex, replace by the encoding you are using
%\usepackage{CJKutf8}                              % if you need to use CJK to typeset your resume in Chinese, Japanese or Korean
\usepackage{multicol}
% adjust the page margins
\usepackage[scale=0.8,top=1cm, bottom=1cm]{geometry}
% \usepackage[scale=0.75]{geometry}
%\setlength{\hintscolumnwidth}{3cm}                % if you want to change the width of the column with the dates
%\setlength{\makecvtitlenamewidth}{10cm}           % for the 'classic' style, if you want to force the width allocated to your name and avoid line breaks. be careful though, the length is normally calculated to avoid any overlap with your personal info; use this at your own typographical risks...
\usepackage{xpatch}
\xpatchcmd\cventry{,}{}{}{}
\usepackage[super]{nth}

\usepackage{xcolor}
\usepackage{geometry}
\definecolor{headerblue}{RGB}{36,92,167} 
% Define header in the preamble
\newcommand{\header}{
    \begin{center}
        {\LARGE \textbf{\textcolor{headerblue}{Negar Honarvar Sedighian}}}\\[0.2cm]
        {\large Shahid Beheshti University, Tehran, Iran}\\[0.2cm]
        \faPhone\ (+98) 903 020 4891 \quad
        \faHome\ \href{https://negarhonarvar.github.io}{Homepage} \quad
        \faEnvelope\ \href{mailto:negarhonarvar.se@gmail.com}{Email} \quad
        \faGithub\ \href{https://github.com/negarhonarvar}{GitHub} \quad
        \faLinkedin\ \href{https://linkedin.com/in/negar-honarvar-sedighian}{LinkedIn}
    \end{center}
}
% \name{Negar}{Honarvar Sedighian}                               % optional, remove / comment the line if not wanted
% \address{70 Absolute Ave.}{L4Z 0A4 Mississauga}{Canada}% optional, remove / comment the line if not wanted; the "postcode city" and and "country" arguments can be omitted or provided empty
\vspace*{3mm}
%comment the line if not wanted
% \phone[fixed]{+2~(345)~678~901}                    % optional, remove / comment the line if not wanted
% \phone[fax]{+3~(456)~789~012}                      % optional, remove / comment the line if not wanted
% \email{negarhonarvar.se@gmail.com }                               % optional, remove / comment the line if not wanted
 % \homepage{linkedin.com/in/negar-honarvar-sedighian }                         % optional, remove / comment the line if not wanted
 % \social[github]{negarhonarvar}   
% \extrainfo{additional information}                 % optional, remove / comment the line if not wanted
%photo[64pt][0.4pt]{picture}                       % optional, remove / comment the line if not wanted; '64pt' is the height the picture must be resized to, 0.4pt is the thickness of the frame around it (put it to 0pt for no frame) and 'picture' is the name of the picture file
% \quote{Some quote}                                 % optional, remove / comment the line if not wanted

% to show numerical labels in the bibliography (default is to show no labels); only useful if you make citations in your resume
%\makeatletter
%\renewcommand*{\bibliographyitemlabel}{\@biblabel{\arabic{enumiv}}}
%\makeatother
%\renewcommand*{\bibliographyitemlabel}{[\arabic{enumiv}]}% CONSIDER REPLACING THE ABOVE BY THIS

% bibliography with mutiple entries
%\usepackage{multibib}
%\newcites{book,misc}{{Books},{Others}}
%----------------------------------------------------------------------------------
%            content
%----------------------------------------------------------------------------------
\begin{document}
%\begin{CJK*}{UTF8}{gbsn}                          % to typeset your resume in Chinese using CJK
%-----       resume       ---------------------------------------------------------
\vspace*{-1.05mm}

\header
\vspace*{-10mm}
% \makecvtitle
\section{Education}
\item{\cventry{Sep. 2020 to Feb. 2025[Expected]}{Bachelor of Science in Computer Engineering}{\href{https://www.sbu.ac.ir/web/cse}{Shahid Behehshti University} }{Tehran, Iran}{}{\vspace{3pt} 
\begin{itemize}
\item \textbf{Cumulative GPA: 17.42/20 (3.67/4)}
\item \textbf{GPA of last two years: 18.23/20 (3.78/4)}
\end{itemize}
}}

{}{Relevant Courses: GPA: 4/4}
\vspace{-1.0em}\begin{small}
 \begin{multicols}{3}
    \begin{itemize}
- Fundamentals of Computer Vision

- Fundamentals of Robotics

- Deep Reinforcement Learning 

- Machine Learning

- Artificial Intelligence

- Algorithms Design

- Computer Simulation

- Statistics and Probability 

- Data Structures 

    \end{itemize}
    \end{multicols}\end{small}
\item{\cventry{Sep. 2017 to Jun. 2020}{High School Diploma in Mathematics}
{\href{https://en.wikipedia.org/wiki/National_Organization_for_Development_of_Exceptional_Talents}{Farzanegan 1 Secondary School}}{Mashhad, Iran}{}{\vspace{3pt} 
\begin{itemize}
    \item \textbf{Diploma GPA: 19.98/20 (4.0/4.0)}
\end{itemize}
}}

\end{itemize}


\section{Research Interests}

\item
\begin{tabular}{>{\raggedright\arraybackslash}p{0.45\linewidth}@{}>{\raggedright\arraybackslash}p{0.1\linewidth}>{\raggedright\arraybackslash}p{0.45\textwidth}@{}} 
     - Applicaton of Deep  Learning Methods in Health Care&& - Graph Neural Networks Methods and Application\\ 
     - Accelerated MRI Reconstruction   && - Image Super Resolution and Denoising\\ 
\end{tabular}
\vspace{3pt}

\section{Research Experience }
\item{\cventry{On Going}{IMAGE PROCESSING AND DISTRIBUTED SYSTEMS LAB}
{B.Sc Thesis}{Shahid Beheshti University}{}{\vspace{3pt} 
\begin{itemize}
    \item Proposing a Dynamic Attentive Graph Neural Network for Cardiac MRI Reconstruction in a cascading manner.
    \item Under Supervision of Dr. Mohsen Ebrahimi Moghaddam.
\end{itemize}
}}


\section{Honors and Awards}

\vspace{4pt}

%\begin{itemize}
\begin{itemize}
\item \textbf{Ranked \nth{2}} among 90 in admission among accepted students in the Computer Engineering department, Shahid Beheshti University, 2023.
\end{itemize}

\begin{itemize}
\item \textbf{Ranked \nth{321}}{in National entrance exam for B.Sc Studies among 160,000 students, 2020.}
\end{itemize}
\begin{itemize}
\item \textbf{Ranked \nth{1}}{{in \textbf{National Organization for Development of Exceptional Talents} Secondary School Entrance Exam in Khorasan Razavi Province}, 2017.}
\end{itemize}



\section{Teaching Assistant Experience}
\begin{itemize}

    \item \textbf{Artificial Intelligence } \hfill Sep. 2024 - Present
\begin{itemize}
        \item Lectured by: Dr. Armin Salimi-Badr

    \end{itemize}
\item \textbf{Discrete Mathematics } \hfill Sep. 2024 - Present
\begin{itemize}
        \item Lectured by: Dr. Farshad Safaei


    \end{itemize}

\item \textbf{Algorithms Design } \hfill Sep. 2024 - Present
\begin{itemize}
        \item Lectured by:Dr. Ramak Ghavamizadeh


    \end{itemize}
    \item \textbf{Software Engineering } \hfill Feb. 2024 - Jul. 2024 (6 mos)
    \begin{itemize}
        \item Lectured by: Dr. Mehran Alidoostnia
    \end{itemize}

    \item \textbf{Technical English} \hfill Sep. 2023 - Jul. 2024 (11 mos)
    \begin{itemize}
        \item Lectured by: Dr. Vahidi Asl

    \end{itemize}

    \item \textbf{Computer Architecture} \hfill Sep. 2023 - Jan. 2024 (5 mos)
    \begin{itemize}
        \item Lectured by: Dr. Rahmati
    \end{itemize}

    \item \textbf{Advanced Programming} \hfill Sep. 2023 - Jan. 2024 (5 mos)
    \begin{itemize}
        \item Lectured by: Dr. Vahidi Asl
    \end{itemize}
\newpage
    \item \textbf{Artificial Intelligence}\hfill Sep. 2023 - Jan. 2024 (5 mos)
    \begin{itemize}
        \item Lectured by: Dr. Mehrnoush Shamsfard
    \end{itemize}

    \item \textbf{Compiler Design} \hfill Sep. 2023 - Jan. 2024 (5 mos)
    \begin{itemize}
        \item Lectured by: Dr. Mehran Alidoostnia
    \end{itemize}

    \item \textbf{Computational Intelligence} \hfill Sep. 2023 - Jan. 2024 (5 mos)
    \begin{itemize}
        \item Lectured by: Dr. Shahabedin Nabavi
    \end{itemize}

    \item \textbf{Operating Systems Labratory}\hfill Sep. 2023 - Jan. 2024 (5 mos)
    \begin{itemize}
        \item Lectured by: Dr. Shahabedin Navabi
    \end{itemize}

\item \textbf{Statistic and Probability}\hfill Sep. 2023 - Jan. 2024 (5 mos)
    \begin{itemize}
        \item Lectured by: Dr. Farshad Safaei

\end{itemize}



\section{Projects}
\cventry{}{{\textbf{\href{https://github.com/negarhonarvar/StockTrading_DRL}{Link to GitHub Repository}}}}{Automated Stock Trading Strategy with DRL}{Jun. 2024}{}{
    \vspace{2pt}
    \begin{itemize}
        \item         Designed a Cascading Long Short-Term Memory Proximal Policy Optimization (PPO) model which uses LSTM layers to capture temporal dependencies in stock data and a PPO algorithm to optimize trading decisions.
        \item The environment is from yfinance library with trading data from Jan. 2009 up to Jun. 2024.
    \end{itemize}
}

\cventry{}{\textbf{\textbf{\href{https://github.com/negarhonarvar/DeepReinforcementLearning}{Link to GitHub Repository}}}}{Deep Reinforcement Learning Algorithms}{May. 2024}{}{
    \vspace{2pt}
    \begin{itemize}
        \item A Complete Collection of Deep RL Famous Algorithms implemented in Gymnasium’s most Popular environments.
        \item Implementation of SARSA and DQN with boltzman in CartPole.
        \item Implementation and comparison of DQN, D3QN, and Enhanced D3QN Agents in Lunar Lander environment.
        \item Implementation of Proximal Policy Optimization algorithm in Swimmer, with clipped objective PPO and adaptive kl PPO agents.
    \end{itemize}
}

\cventry{}{ \textbf{\href{https://github.com/negarhonarvar/Enhanced-News-Classification-On-Large-Dataset}{Link to GitHub Repository}}}{Enhanced Farsi News Classification}{Mar. 2024}{}{
    \vspace{2pt}
    \begin{itemize}
        \item The goal of this project is to develop an enhanced neural network model to classify Farsi news articles into their respective categories.
        \item The dataset has been preprocessed with Tokenization and Feature Extraction.
    \end{itemize}
}

\cventry{}{\textbf{ \textbf{\href{https://github.com/negarhonarvar/Computer-Vision/blob/main/README.md}{Link to GitHub Repository}}}}{Classic Computer Vision}{Feb. 2024}{}{
    \vspace{2pt}
    \begin{itemize}
        \item Application of Classic Computer Vision Techniques such as Filtering, Transformation, and Feature Extraction for image interpretation.
    \end{itemize}
}

\cventry{}{\textbf{ \textbf{\href{https://github.com/negarhonarvar/Guidance-of-a-Quadcopter-for-Object-Detection}{Link to GitHub Repository}}}}{Guidance of a Quadcopter for Object Detection}{Mar. 2024}{}{
    \vspace{2pt}
    \begin{itemize}
        \item Designed a controller for a quadcopter to control its flight over boxes in an urban environment, automatically taking precise images of boxes and interpreting the images using Computer Vision Deep Learning-Based Approaches.
        \item After interpreting the image, the quadcopter determines whether the item matches the target item; if matched, the quadcopter lands beside the box and turns on its front LEDs.
    \end{itemize}
}

\cventry{}{\textbf{\href{https://github.com/negarhonarvar/Bug-Algorithms}{Link to GitHub Repository}}}{Bug Algorithms}{Jan. 2024}{}{
    \vspace{2pt}
    \begin{itemize}
        \item Implementation of Bug1, Bug2, and Wall-following algorithms for GCTronic's e-puck in the Webots environment.
        \item Each algorithm successfully guides the robot through a maze.
        \item Map of the maze is generated with Bug2 and split-and-merge algorithms.
    \end{itemize}
}

\cventry{}{ \textbf{\href{https://github.com/negarhonarvar/Machine-Learning}{Link to GitHub Repository}}}{Machine Learning Algorithms}{Jan. 2024}{}{
    \vspace{2pt}
    \begin{itemize}
        \item This repository includes famous classification and regression algorithms, each applied to solve a related problem.
        \item Each problem includes Feature Engineering methods to prepare raw data by transforming it into relevant features.
        \item Algorithms include K-Nearest Neighbors, Support Vector Machine (SVM), Decision Tree, and Gradient Descent for supervised learning; DBSCAN is used as an unsupervised algorithm.
    \end{itemize}
}

\cventry{}{ \textbf{\href{https://github.com/negarhonarvar/Robatics}{Link to GitHub Repository}}}{Robotics}{Dec. 2023}{}{
    \vspace{2pt}
    \begin{itemize}
        \item Controllers for e-puck in Webots environment using popular Localization, Planning, and Navigation algorithms.
        \item The controllers range from simple to complex, providing beginners with a better understanding of the control process.
    \end{itemize}
}
\newpage
\cventry{}{ \textbf{\href{https://github.com/negarhonarvar/TronGame}{Link to GitHub Repository}}}{Tron Game Agent}{May. 2023}{}{
    \vspace{2pt}
    \begin{itemize}
        \item The algorithm devised for this game is a combination of a Genetic Algorithm and Minmax, where Minmax is used as the fitness function for the Genetic Algorithm.
        \item This game consists of two real-time agents that try to create more walls than their opponent while avoiding collisions with each other and the boundary walls. The Unity framework used is based on Chillin’s monitor games.
    \end{itemize}
}

\section{\textbf{Referrers}}

\vspace{4pt}

%\begin{itemize}
\begin{itemize}
\begin{itemize}
    
    \item \textbf{Dr. Armin Salimi-Badr} \\
    \hfill Assistant Professor of Software Engineering, \href{https://scholar.google.com/citations?view_op=view_org&hl=en&org=15042406923477825387}{Shahid Beheshti University}
        \item \textbf{Dr. Shahabedin Nabavi} \\
    \hfill Faculty of Computer Science and Engineering, \href{https://scholar.google.com/citations?view_op=view_org&hl=en&org=15042406923477825387}{Shahid Beheshti University}
    
    \item     \textbf{Dr. Mojtaba Vahidi-Asl} \\
    \hfill Assistant Professor of Software Engineering, \href{https://scholar.google.com/citations?view_op=view_org&hl=en&org=15042406923477825387}{Shahid Beheshti University}
    \item \textbf{Dr. Mehran Alidoost Nia} \\
    \hfill Assistant Professor of Computer Engineering, \href{https://scholar.google.com/citations?view_op=view_org&hl=en&org=15042406923477825387}{Shahid Beheshti University}
\end{itemize}
\end{itemize}


  
  
\end{document}